\chapter[Conclusao]{Conclusão}

Este trabalho apresentou um estudo da implementação em sistema coprocessado hardware-software do algoritmo de treinamento do classificador LDA, comparadando os resultados obtidos com outras duas implementações em arquiteturas diferentes, sendo elas a implemetação em \textit{Matlab} \cite{F.Lotte} e a implementação em linguagem C.\\

De acordo com os resultados obtidos em simulação a expectativa era de se obter um menor tempo de processamento na implementação coprocessada, entretanto nos resultados simulados não se fez presente o tempo de tráfego de dados no barramento \textit{AXI-4Lite}, entre o processador (ARM) e a lógica programável (IPs). Com isso a partir dos resultados da Tabela \ref{tempos} é possível concluir que o sistema coprocessado não é o mais eficiente em tempo de execução, ficando atrás da implementação em linguagem C, mas ainda assim mais eficiente que a implementação em \textit{Matlab}. Isso pode ser decorrente do ambiente de execução da implementação em C, que foi executada sobre o auxilio de um sistema operacional Linux (versão para desenvolvedores), sendo esta com uma quantidade de processos reduzidas, consequentemente consome menos recurso de processamento. Sendo assim o objetivo desde presente trabalho foi
alcançado, mesmo apresentando resultados inesperados.\\


Este trabalho mostrou que é possível aplicar métodos de processamento em arquitetura de hardware reconfigurável aplicado à classificação de sinais. Portanto, para trabalhos futuros, pode ser feito uma melhor interface entre hardware e software para obter um melhor tempo de execução, pois os resultados deste trabalho mostram que o barramento \textit{AXI-4Lite} foi um limitante para se ter os resultados esperados, além dos cálculos realizados em software apresentarem um desempenho computacional superior às outras implementações.

