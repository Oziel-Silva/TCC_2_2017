\chapter[Introdução]{Introdução}


\section{Contextualização}

Através de uma rede de mais de 100 bilhões de células nervosas interconectadas, o cérebro
realiza o controle de nossas ações, percepções, emoções e etc \cite{KANDEL}. Estas células
são chamadas de \textit{neurônios}, e neles são armazenados sinais elétricos, que representam
todas as informações de controle \cite{Siulybook}. Estes sinais podem ser medidos pela
eletroencefalografia (EEG), que é um sistema de medição de sinais elétricos produzidos pelo
cérebro durante atividades cerebrais \cite{F.Lotte}. Segundo \cite{SIULYDissertacao}, a EEG
é uma das mais importantes ferramentas para diagnosticar doenças cerebrais.

Além do diagnóstico de doenças cerebrais uma outra aplicação para os sinais adquiridos
pela EEG são as \textit{Brain Computer Interfaces} (BCIs) \cite{F.Lotte}. Uma BCI é um
sistema que realiza a comunicação entre o cérebro e um computador \cite{Siulybook}, onde
sua principal função é a tradução dos sinais elétricos, obtidos através da EEG, em comandos
de controle para qualquer dispositivo eletrônico \cite{Siulybook}.

A BCI realiza a tradução destes comandos através de seis passos: 1) medição dos sinais
provenientes de atividades cerebrais através da EEG, 2) pré-processamento destes sinais,
3) extração de características, 4) classificação, 5)tradução dos sinais em comandos e 6)
realimentação \cite{MasonAndBirch}. Um dos principais passos para a implementação de uma
BCI é a \textbf{classificação}, pois é após este passo que é realizada a tradução dos sinais provenientes da EEG
em comandos de controle \cite{MasonAndBirch}.

A classificação de um sinal é caracterizada, em aprendizado de máquina e em reconhecimento de
padrões, como um algoritmo que atribui parte de um dado sinal de entrada a um dado número de
classes ou categorias \cite{brunelli2009template}. Um exemplo é a classificação de um e-mail
como "spam" ou "não-spam". Os algoritmos que realizam a classificação dos sinais de entrada são
chamados de \textbf{classificadores} \cite{Siulybook}. De acordo com \cite[p. 41]{lottephd}, "estes
classificadores são capazes de aprender como identificar um vetor de características, graças
aos processos de treinamentos". Estes conjuntos são formados por vetores de características
previamente atribuídos às suas respectivas classes \cite{lottephd}.

O algoritmo que realiza a classificação é caracterizado por uma função matemática
que mapeia um sinal de entrada em sua respectiva classe \cite{lottephd}. Os classificadores
preferidos pelos pesquisadores são os \textbf{classificadores supervisionados}, pois estes
tipos de classificadores necessitam de um conjunto de dados de treinamento. Os dados de treinamento são um conjunto de dados previamente classificados \cite{Siulybook}. Portanto os classificadores supervisionados são implementados a partir de dois processos: \textit{treinamento} e \textit{testes} \cite{Siulybook}. As \textit{Support Vector Machines} (SVM), os \textit{Linear Discriminant Analysis} (LDA), os filtros Kalman, as árvores de decisões são alguns exemplos de classificadores do tipo supervisionados \cite{Siulybook}.

O LDA como dito anteriormente é um dos classificadores supervisionados e tem como suas principais vantagens
a simplicidade e atratividade computacional, por se tratar de um classificador linear \cite{patternRecogn}.
O objetivo do LDA é usar uma transformação linear para encontrar um conjunto otmizado de vetores
discriminantes e remapear o conjunto de características
original, em um outro conjunto de dimensão  inferior \cite{ShashoaLDA}.

Apesar do LDA ser um algorítmo interessante no que se trata de consumo computacional, em geral, os algoritmos de classificação são complexos computacionalmente com tempo de execução elevado,
isso os tornam restritivos a poucas aplicações, para executar um 
códido que descreve esses algoritmos é necessário uma máquina de proporções não versátil e não portátil,
assim não podendo equipar projetos que têm restrições de dimensão e peso.

Um \textit{System on Chip} (SoC) é caracterizado pela implementação de todo um sistema computacional, tais como memórias, processadores, entradas e saídas, lógicas digitais, entre outros, em um único chip de silício \cite{zynqBook}. Diferente dos computadores tradicionais, que possuem seu sistema implementado a partir de módulos isolados e combinados em uma placa de circuito impresso, ou placa-mãe, os SoCs possuem como principais características um baixo custo de implementação, além de baixo consumo de potência, menor tamanho físico, maior confiabilidade e maior velocidade do sistema geral, quando comparado com um computador tradicional \cite{zynqBook}. Um exemplo de um SoC é a plataforma Zynq que combina em um único chip processadores \textit{Advanced Risc Machine} (ARM) e \textit{Field Programmable Gate Array} (FPGA), esse útimo utilizado para configurar todos os módulos de um computador tradicional \cite{zynqBook}. 

Tendo em vista a grande vantagem dos SoCs sobre os computadores tradicionais, onde são implementados e executados os algoritmos de classificação, este trabalho apresenta um estudo da viabilidade da implementação em hardware e em software embarcado, do algoritmo de treinamento do classificador LDA, realizando a comparação de consumo computacional, processamento computacional (tempo de execução), entre as implementações em hardware e software, onde a implementação em hardware consiste no mapeamento do algoritmo na plataforma FPGA, afim de paralelizar seus processos e a implementação em software consiste em executar o algoritmo em um sistema embarcado utilizando os cores ARM, ambos da plataforma Zynq, além da comparação com sua implementação inicial em \textit{Matlab}.


\section{Justificativa}
As aplicações das BCIs apresentam um crescente desenvolvimento, graças ao aumento do interesse em pesquisas voltadas para o tema (BCI COMPETITION). Por ser considerado o principal processo das BCIs, a classificação requer um cuidado especial \cite{MasonAndBirch}. Como o LDA é um classificador supervisionado, a acurácia da classificação depende inteiramente de um bom treinamento \cite{F.Lotte}. Isso requer do algoritmo de treinamento um maior esforço computacional. Como os SoCs apresentam características de baixo consumo de potência, tamanho físico pequeno e possuir todos módulos de um sistema computacional em um único chip, a implementação de algoritmos de classificação em sistema deste porte podem ou não tornar as BCIs mais acessíveis, tendo em vista que um algoritmo de treinamento embarcado em um SoC reduzirá a necessidade de um sistema computacional tradicional, além de um melhor processamento computacional, pois sua implementação em FPGA na plataforma \textit{Zynq} torna-se possível paralelizar seus processos de execução \cite{zynqBook}.

\section{Objetivos}
Esta seção apresenta os objetivos gerais e específicos propostos a serem desenvolvidos neste
 presente trabalho.
\subsection{Objetivos Gerais}
\begin{itemize}
	\item Implementar parte do algoritmo de treinamento de um classificador LDA utilizando um SoC na plataforma Zynq afim de otimizar tanto o algoritmo, em relação a tempo de execução, quanto o seu consumo de recursos.
\end{itemize} 
\subsection{Objetivos Específicos}

\begin{itemize}
	\item Explorar o algoritmo de treinamento do classificador LDA desenvolvido por \cite{F.Lotte};
	\item Mapear parte deste algoritmo em arquiteturas paralelas utilizando a linguagem VHDL;
	\item Implementar em sistema embarcado o algoritmo de treinamento utilizando os cores ARM da Zynq;
	\item Realizar teste e validação das implementações utilizando as bases de dados do BCI Competition III, em especifico o conjunto de dados BCI III \textit{dataset} IVa.
\end{itemize}
