\chapter[Introdução]{Introdução}

Através de uma rede de mais de 100 bilhões de células nervosas interconectadas, o cérebro realiza o controle de nossas ações, percepções, emoções e etc \cite{KANDEL}. Estas células são chamadas de \textit{neurônios}, e neles são armazenados sinais elétricos, que representam todas as informações de controle \cite{Siulybook}. Estes sinais podem ser medidos pela eletroencefalografia (EEG), que é um sistema de medição de sinais elétricos produzidos pelo cérebro durante atividades cerebrais \cite{F.Lotte}. 

De acordo com \cite{SIULYDissertacao}, a EEG é uma das mais importantes ferramentas para diagnosticar doenças cerebrais. Além do diagnóstico de doenças cerebrais uma outra aplicação para os sinais adquiridos pela EEG são as \textit{Brain Computer Interfaces} (BCIs) \cite{F.Lotte}. Uma BCI é um
sistema que realiza a comunicação entre o cérebro e um computador \cite{Siulybook}, onde sua principal função é a tradução dos sinais elétricos cerebrais em comandos de controle para qualquer dispositivo eletrônico \cite{Siulybook}.

A BCI realiza a tradução destes sinais através de seis passos: 1) medição dos sinais provenientes de atividades cerebrais, normalmente através da EEG, 2) pré-processamento destes sinais, 3) extração de características, 4) classificação, 5)tradução dos sinais em comandos e 6) realimentação \cite{MasonAndBirch}. Um dos principais passos para a implementação de uma
BCI é a \textit{classificação}, pois é após este passo que é realizada a tradução dos sinais provenientes da EEG em comandos de controle \cite{MasonAndBirch}.

Em aprendizado de máquina e reconhecimento de padrões, a classificação é caracterizada como um algoritmo que atribui parte de um sinal de entrada a um número de classes ou categorias \cite{brunelli2009template}. Um exemplo é a classificação de um e-mail como spam ou não-spam. Os algoritmos que realizam a classificação dos sinais de entrada são chamados de \textbf{classificadores} \cite{Siulybook}. De acordo com \cite[p. 41]{lottephd}, "estes classificadores são capazes de aprender como identificar um vetor de características, graças aos processos de treinamento".

O algoritmo que realiza a classificação é caracterizado por uma função matemática que mapeia um sinal de entrada em sua respectiva classe \cite{lottephd}. Os classificadores preferidos pelos pesquisadores são os \textbf{classificadores supervisionados}.Estes classificadores fazem uso de um conjunto de dados para realizar o processo de treinamento do classificador. O conjunto de dados de treinamento é formado por vetores de características previamente atribuídos às suas respectivas classes \cite{lottephd}. Portanto os classificadores supervisionados são implementados a partir de dois processos: \textit{treinamento} e \textit{testes} \cite{Siulybook}. As \textit{Support Vector Machines} (SVMs), em português, Máquinas de Vetor de Suporte, os \textit{Linear Discriminant Analysis} (LDAs), em português, Análise Discriminante Linear, as Redes Neurais Artificiais (RNAs) e as árvores de decisões são alguns exemplos mais conhecidos de classificadores do tipo supervisionados \cite{Siulybook}.

O LDA como dito anteriormente é um dos classificadores supervisionados e tem como principais vantagens a simplicidade e atratividade computacional, por se tratar de um classificador linear \cite{patternRecogn}. O objetivo do LDA é usar uma transformação linear para encontrar um conjunto otmizado de vetores
discriminantes e remapear o conjunto de características original, em um outro conjunto de dimensão  inferior \cite{ShashoaLDA}.

A simplicidade e robustez do LDA possibilita sua implementação em um sistema embarcado, o que viabiliza sua portabilidade considerando as restrições de desempenho computacional e consumo energético dos Sistemas em Chip (SoCs - do inglês \textit{System on Chip}.

Um SoC é caracterizado pela implementação de todo um sistema computacional, composto por: memórias, processadores, entradas e saídas, conversores de dados, controladores de periféricos, entre outros, em um único chip de silício \cite{zynqBook}. Diferente dos computadores tradicionais, que possuem seu sistema implementado a partir de módulos isolados e combinados em uma placa de circuito impresso, ou placa-mãe, os SoCs possuem como principais características um baixo custo de implementação, além de baixo consumo de potência, menor tamanho físico, maior confiabilidade e, dependendo dos recursos disponíveis, maior desempenho computacional, se comparado com um computador tradicional \cite{zynqBook}. Um exemplo de um SoC é a plataforma \textit{Zynq} que combina em um único chip processadores \textit{Advanced Risc Machine} (ARM) e \textit{Field Programmable Gate Arrays} (FPGA), conversores analógicos digitais (ADC - do inglês) \cite{zynqBook}. 

Tendo em vista a grande vantagem dos SoCs sobre os computadores tradicionais, onde são implementados e executados os algoritmos de classificação, este trabalho apresenta um estudo da viabilidade da implementação em hardware e em software embarcado, do algoritmo de treinamento do classificador LDA desenvolvido previamente por \cite{F.Lotte} na plataforma \textit{Matlab}, realizando a comparação de consumo computacional (hardware), desempenho computacional (tempo de execução) e consumo energético entre as implementações em hardware e software. Em particular a implementação em hardware consiste no mapeamento do algoritmo na FPGA da plataforma \textit{Zynq}, afim de paralelizar seus processos, enquanto que a implementação em software consiste em executar o algoritmo em um sistema embarcado utilizando os cores ARM, também da plataforma \textit{Zynq}, além da comparação com sua implementação inicial em \textit{Matlab}.


\section{Justificativa}
As aplicações das BCIs apresentam um crescente desenvolvimento  graças ao aumento do interesse em pesquisas voltadas para o tema \cite{BCICompetition} . As BCIs utilizam-se de algoritmos de classificação para realizar o processo de tradução dos sinais cerebrais em comandos de controle \cite{MasonAndBirch}. O LDA é um tipo de classificador utilizado para realizar tal processo. Um dos principais processos para implementação de um classificador é o processo de treinamento, é com este processo que o classificador é capaz de classificar corretamente os sinais de entrada, quando realizado um bom processo de treinamento\cite{F.Lotte}. Por se tratar de um algoritmo linear e consequentemente não exigir um grande esforço computacional, torna-se viável sua implementação em um sistema embarcado. Como os SoCs apresentam características de baixo consumo de energia e tamanho físico pequeno, a implementação de algoritmos de classificação em sistema deste porte podem tornar as BCIs mais acessíveis, tendo em vista que um algoritmo de treinamento embarcado em um SoC reduzirá a necessidade de um sistema computacional tradicional, além de um melhor processamento computacional, pois sua implementação em plataformas FPGAs possibilitam a paralelização de seus processos \cite{zynqBook}.

\section{Objetivos}

\subsection{Objetivos Gerais}

	O objetivo geral do presente trabalho é implementar o algoritmo de treinamento de um classificador LDA em um sistema embarcado coprocessado (hardware/software), utilizando um SoC na plataforma \textit{Zynq} da \textit{Xilinx}, no intuito de estudar os ganhos em relação ao tempo de execução se comparado com implementações tradicionais.

\subsection{Objetivos Específicos}

\begin{itemize}
	\item Reproduzir os resultados obtidos por \cite{F.Lotte} no desenvolvimento do algoritmo de treinamento do classificador LDA;

	\item Implementar em sistema embarcado o algoritmo de treinamento utilizando os cores ARM da \textit{Zynq};
	
	\item Realizar uma análise de \textit{profile} para determinação de funções que exigem um maior esforço computacional;
	
	\item Mapear a(s) função(ões) que exigem um maior esforço computacional em hardware FPGA utilizando a linguagem VHDL;
	

	\item Validar as implementações utilizando as bases de dados do \textit{BCI Competition III}, em especifico o conjunto de dados BCI III \textit{dataset IVa}.
	
	\item Realizar uma análise estatística do erro associado a ambas implementações, comparadas com a implementação na plataforma \textit{Matlab}.
	
	\item Realizar uma análise de ganhos ou perdas de processamento computacional, comparando a implementação em software com o coprocessamento;
\end{itemize}
