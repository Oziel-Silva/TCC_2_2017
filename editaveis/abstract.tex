\begin{resumo}[Abstract]
 \begin{otherlanguage*}{english}

   \vspace{\onelineskip}
 	\textit{The Brain Computer Interface (BCI) are systems capable of making a communication between the brain and electronical devices. As they are scientifically studied more and more, BCIs already present a big number of applications. One of the main principles of implementation of a BCI is the classification of the signals generated by the brain and starting from the classification that the processes of commands are executed. There are numerous algorithms that perform this type of classification, one of them is the Linear Discriminant Analysis classifier (LDA). In 2010 the French scientist Fabien Lotte published a work in which realizes the implementation of this classifier, obtaining as best result of accuracy 96.43\% in the classification of signals of motor imagery provided by the BCI Competition III. One of the important points and the greater processing need to implement this classifier is a process of training, in which the hyperplanes capable of separating classes from the signals in study are obtained. These hyperplanes are obtained through matrix calculations. One of the systems able to accelerate algorithms that perform this type of calculation are System on Chip (SoC) that contain FPGA, in which the parallelism of processes is explored. Therefore, in this work it is presented a study of the implementation in floating-point calculations of the algorithm of training of the LDA classifier in a hardware-software co-processed system using the Zynq-7000 SoC system (consisting of an ARM Cortex A9 dual core processor and a FPGA Artix-7).  In which it compares with implementations in Matlab developed by Fabien Lotte and the implementation of a embedded system using C programming language. The results showed that the algorithm implemented in C language presented better computational performance of the order of 93 times faster than the algorithm executed in Matlab. The co-processed system performs better in computing functions because of its parallelism. However, the system communication bus latency in hardware with the software system is a limitation of its performance, presenting speed approximately 8 times faster than the Matlab implementation. In addition, implementations in C and co-processed Language presented a energy consumption approximately 7 times lower than the traditional computer. Thus, the best implementation for this algorithm is the C language.}
 	
   \noindent 
   \textbf{Key-words}: BCI; LDA; FPGA; SoC; Embedded Systems.
 \end{otherlanguage*}
\end{resumo}
