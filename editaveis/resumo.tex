\begin{resumo}

As \textit{Brain Computer Interfaces} (BCI) são sistemas capazes de realizar uma comunicação entre o cérebro e dispositivos eletrônicos. Cada vez mais estudadas por cientistas da área as BCIs já apresentam um grande número de aplicações. Um dos principais procedimentos para implementação de uma BCI é a classificação dos sinais gerados pelo cérebro. E a partir da classificação que os processos de comandos são executados. Existem inúmeros algoritmos que realizam este tipo de classificação, um deles é o classificador \textit{Linear Discriminant  Analysis} (LDA). Em 2010 o cientista francês Fabien Lotte publicou um trabalho onde realiza a implementação deste classificador, na plataforma \textit{Matlab}, obtendo como melhor resultado de acurácia 96,43\% na classificação de sinais de imagética motora, fornecidos pela competição \textit{BCI Competition III}. Um dos pontos importantes e de maior necessidade de processamento para implementação deste classificador é processo de treinamento, onde são obtidos os hiperplanos capazes de separar as classes dos sinais em estudo. Estes hiperplanos são obtidos através de cálculos matriciais. Um dos sistemas capazes acelerar algoritmos que realizam este tipo de cálculo são os SoCs que contenham FPGA, onde são explorados o paralelismo de processos.
Sendo assim, neste trabalho é apresentado um estudo da implementação em cálculos de ponto flutante do algoritmo de treinamento do classificador LDA em um sistema coprocessado hardware-software utilizando-se do SoC \textit{Zynq-7000} (onde se encontram embarcado um processador \textit{ARM Cortex A9} dual core e um FPGA \textit{Artix-7}), realizando a comparação com implementações em \textit{Matlab} desenvolvida por Fabien Lotte e a implementação em um sistema embarcado utilizando Linguagem de porgamação C. Os resultados mostraram que o algoritmo implementado em linguagem C apresentou melhor desempenho computacional sendo ele 17 vezes mais rápido que o algoritmo executado em \textit{Matlab}. Já o sistema coprocessado apresenta um melhor desempenho em funções de cálculo devido ao seu paralelismo, entretanto a latência do barramento de comunicação do sistema em hardware com o sistema em software é um limitante do seu desempenho. Sendo assim a melhor implementação para este algoritmo é a linguagem C.
 \vspace{\onelineskip}
 
 \noindent
 \textbf{Palavras-chaves}: BCI; LDA; FPGA; SoC; Sistemas Embarcados.
\end{resumo}
