\begin{resumo}

 \vspace{\onelineskip}
 As aplicações com as \textit{Brain Computer Interfaces} (BCI) apresentam um desenvolvimento crescente graças ao aumento do interesses de pesquisadores sobre o tema. Um dos principais passos para desenvolvimento de uma BCI é a classificação dos sinais cerebrais, que posteriormente são convertidos em comandos de controle para um dispositivo. Alguns classificadores apresentam características de linearidade, não exigindo um alto esforço computacional para sua execução, o que possibilita a sua implementação em um sistema embarcado. Sendo assim, este trabalho apresenta um estudo comparativo entre as implementações do algoritmo de treinamento  do classificador \textit{Linear Discriminant Analysis} (LDA), em hardware e em software, utilizando o SoC da família \textit{Zynq} embarcado no kit de desenvolvimento \textit{Zybo-board}.
 \noindent
 \newline
 \textbf{Palavras-chaves}: BCI; LDA; FPGA; SoC; Sistemas Embarcados.
\end{resumo}
