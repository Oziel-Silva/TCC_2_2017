\begin{resumo}

As Interfaces Cérebro-Máquina (BCI, do inglês \textit{Brain Computer Interfaces} são sistemas capazes de realizar uma comunicação entre o cérebro e dispositivos eletrônicos. Cada vez mais estudadas no âmbito científico as BCIs já apresentam um grande número de aplicações. Um dos principais procedimentos para implementação de uma BCI é a classificação dos sinais gerados pelo cérebro, pois é após a classificação que os processos de comandos são executados. Existem vários algoritmos que realizam este tipo de classificação, um deles é o classificador \textit{Linear Discriminant  Analysis} (LDA). Em 2010 o cientista francês Fabien Lotte publicou um trabalho no qual realiza a implementação deste classificador, obtendo como melhor resultado de acurácia 96,43\% na classificação de sinais de imagética motora, fornecidos pela competição \textit{BCI Competition III}. Um dos pontos importantes e de maior necessidade de processamento para implementação deste classificador é processo de treinamento, nos quais são obtidos os hiperplanos capazes de separar as classes dos sinais em estudo. Um dos sistemas capazes acelerar algoritmos que realizam este tipo de cálculo são os SoCs que contêm FPGA, nos quais são explorados os paralelismos de processos.
Sendo assim, neste trabalho é apresentado um estudo da implementação em cálculos de ponto flutante do algoritmo de treinamento do classificador LDA em um sistema coprocessado hardware-software utilizando o Sistema em Chip (SoC, do inglês \textit{System on Chip}) \textit{Zynq-7000} (composto por  um processador \textit{ARM Cortex A9} dual core e um FPGA \textit{Artix-7}). Esta implementação é comparada com a implementação em \textit{Matlab} desenvolvida por Fabien Lotte e a implementação em um sistema embarcado utilizando Linguagem de programação C. Os resultados mostraram que o algoritmo implementado em linguagem C apresentou melhor desempenho computacional da ordem de 93 vezes mais rápido que o algoritmo executado em \textit{Matlab}. Já o sistema coprocessado apresenta um melhor desempenho em funções de cálculo devido ao seu paralelismo. Entretanto a latência do barramento de comunicação do sistema em hardware com o sistema em software é um limitante do seu desempenho, apresentando velocidade de processamento de aproximadamente 8 vezes mais rápido que a implementação em \textit{Matlab}. Além disso, as implementações em linguagem C e sistema coprocessado apresentaram um consumo energético de aproximadamente 7 vezes menor que a implementação em um computador tradicional.


 \vspace{\onelineskip}
 
 \noindent
 \textbf{Palavras-chaves}: BCI; LDA; FPGA; SoC; Sistemas Embarcados.
\end{resumo}
