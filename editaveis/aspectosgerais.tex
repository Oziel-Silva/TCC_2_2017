
\chapter[Referencial Teórico]{Referencial Teórico}

\section{O Cérebro}
O SNC (Sistema Nervoso Central) é o responsável direto pelo comando do nosso comportamento geral\cite{David_Clarck}. Ele pode ser dividido em duas principais áreas: medula espinhal e o cérebro \cite{KANDEL}.
A medula espinhal, parte caldal do SNC recebe e processa todos os sinais dos sensores corporais, além de realizar o controle dos membros e do tronco humano \cite{KANDEL}.
O cérebro é o processador central do SNC, nele são recebidos e processados os sinais da medula espinhal, além de fornecer todos os sinais de controle para a própria medula \cite{KANDEL}. O cérebro é dividido em 3 principais regiões: cerebrum, cerebelo e tronco encefálico \cite{SIULYDissertacao}.
\section{Eletroencefalografia}

\section{\textit{Brain Computer Interface}}

\section{\textit{BCI Competition}}
A \textit{BCI Competition} é uma competição que promove o desenvolvimento e melhoria da tecnologia voltada para as BCIs, onde são submetidas diferentes técnicas de análise de dados cerebrais \cite{BCICompetition}. Já foram realizadas quatro edições da competição, nos anos de 2001, 2002, 2004 e 2008 \cite{BCICompetition}. Em cada uma destas competições são fornecidos publicamente sinais cerebrais, adquiridos em laboratórios especializados \cite{BCICompetition}.
 
\section{\textit{Linear Discriminant Analisys}}
Supondo a existência de um conjunto de dados L,com características multivariadas, e que cada dado
seja conhecido devido ser proveniente de uma das  classes K, tal que são predefinidas com características
semelhantes aos dados. As classes podem ser exemplificadas como sendo: espécies de plantas,
precença ou ausência de uma condição médica específica, diferentes tipos de tumores, tipos de veículos automotores
entre outros. Para separar as classes conhecidas uma das outras, é atribuido um rótulo a cada classe, então os dados são
representados como dados rotulados.(livro Modern Multivariate
Statistical Techiniques -Alan J. Izenman)


Devido a indispensabilidade de diminuir as dimensões dos dados de um determinado conjunto, o objetivo do LDA
é reduzir a dimensão do espaço de conjunto de dados, resolvendo o inconvêniente da sobreposição.  
(A Comparison of Linear Discriminant Analysis and Ridge Classifier on Twitter Data)



\section{\textit{System-on-Chip}}

\textit{System-on-Chip} (SoC), implica que todo sistema que
contém funcionalidades implementadas em \textit{hardware}
e \textit{software} se encontra em um único chip de silício,combinando processamento, lógica de
alta velocidade, interface, memória entre outros componentes ao invés de 
uma implementação maior em vários \textit{chips} físicos diferentes agrupados em uma placa 
de cicuito impresso.(livro zynq book)


São vários os argumentos a favor da escolha de um SoC a uma placa de circuito impresso, pode-se
citar que a solução é de menor custo, viabiliza transferência de dados mais rápidas e seguras entre
vários elementos do sistema, possui maior velocidade geral do sistema, menor consumo de energia entre
vários outros elementos que fortalecem a escolha de um SoC em sistemas discretos com componentes
equivalentes. (livro zynq book)


\section{Estado da Arte}



