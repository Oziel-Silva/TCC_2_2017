
\chapter[Referencial Teórico]{Referencial Teórico}

\section{O Cérebro}
O SNC (Sistema Nervoso Central) é o responsável direto pelo comando do nosso comportamento geral\cite{David_Clarck}. Ele pode ser dividido em duas principais áreas: medula espinhal e o cérebro \cite{KANDEL}.
A medula espinhal, parte caldal do SNC recebe e processa todos os sinais dos sensores corporais, além de realizar o controle dos membros e do tronco humano \cite{KANDEL}.
O cérebro é o processador central do SNC, nele são recebidos e processados os sinais da medula espinhal, além de fornecer todos os sinais de controle para a própria medula \cite{KANDEL}. O cérebro é dividido em 3 principais regiões: cerebrum, cerebelo e tronco encefálico \cite{SIULYDissertacao}.
\section{Eletroencefalografia}

\section{\textit{Brain Computer Interface}}

\section{\textit{BCI Competition}}

\section{\textit{Linear Discriminant Analisys}}

\section{Sistemas em Chip (SoC)}

\section{Estado da Arte}



